%%%%%%%%%%%%%%%%%%%%%%%%%%%%%%%%%%%%%%%%%%%%%%%%%%%%%%%%%%%%%%%%%%%
%                                                                 %
%                            ABSTRACT                             %
%                                                                 %
%%%%%%%%%%%%%%%%%%%%%%%%%%%%%%%%%%%%%%%%%%%%%%%%%%%%%%%%%%%%%%%%%%%

\specialhead{ABSTRACT}

Establishing a population estimate for the plains zebras and Masai giraffes in the Nairobi National Park (NNP) can be a challenging task due to the large size of the conservation area and large population.  Making the situation worse, the NNP is not fenced on its southern boundary, which makes traditional counting methods impractical and unpredictable due to the park's arbitrary population.  Traditional and invasive identification methods (e.g.\ ear tags, ear notches, radio collars) are costly and infeasible for large populations.  As an alternative, we propose a passive, appearance-based approach that uses images of animals taken by volunteer ``citizen scientists'' to identify individuals.  Image data is analyzed using our prototype IBEIS computer vision algorithm, which recognizes animals based solely on their appearance.  The collection of images over time allows for a more comprehensive ecological analysis of the animal population and the park's ecosystem.   By providing actionable ecological data, our method allows the conservationists in the NNP to make data-driven decisions in order to accomplish their conservation goals.

In March of 2015, the IBEIS team helped to administer \textit{The Great Zebra \& Giraffe Count} (GZGC), which collected 9,406 images from 58 volunteer citizen scientists.  The contributed images yielded a total of 8,659 sightings of plains zebras (\textit{Equus quagga}), of which we identified 1,258 individuals.  We performed a \textit{photographic censusing}, or photographic mark-recapture (PMR) study, on the last two days of data collection and, using the Lincoln-Peterson method, estimated a total of $2,307 \pm 366$ zebras in the NNP (confidence of 95\%).  The IBEIS team also analyzed images of Masai giraffes (\textit{Giraffa camelopardalis tippelskirchi}), which yielded a total of 1,258 sightings.  Of the sighted giraffes, we found 103 individuals and calculated a Peterson-Lincoln Index of $119 \pm 48$.

To our knowledge, this is the first time a population estimate of the plains zebras and Masai giraffes has ever been performed using an automated appearance-based approach.  We also believe that the GZGC was the largest citizen science data collection event ever performed inside the park, to date.
